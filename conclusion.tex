In conclusion the mass's of the \textUpsilon(1s), \textUpsilon(2s) and \textUpsilon(3s) mesons were determined to be 9.4(3), 10.0(3) and 10.3(3) $[GeV/c^2]$. The masses of the mesons were determined by binning the data using the Freedman Diaconis method and the resulting histogram was fitted using a Probability Density Function. This PDF was the sum of three double Gaussian's and an exponential decay. Of the different models that were tried, the double Gaussian was most successful in fitting the data overall, but was unable to correctly fit the asymmetry of the \textUpsilon(1s) peak. The fitting function that was able to best describe the asymmetry was the Crystal Ball, but it fell short when fitting the amplitude of the \textUpsilon(1s) peak. The errors were determined by combining statistical errors and systematic errors. For future work, one could fit the histogram using double Crystal Ball functions instead of the single Crystal Ball's. This was not done due to a lack of time.