\addcontentsline{toc}{subsection}{Program Description}
\subsection*{Program Description}
Initially, all curve fitting was done using SciPy's \verb|curve_fit()| method, however this was unsuccessful in fitting the Crystal Ball (see figure 13 in appendix). The main issues arose when trying to fit the second and third peak. The normalisation of the Crystal Ball (see equation 6) depends on the values of \textalpha\ and n. It is likely that the Crystal Ball used, particularly the normalisation of this function, was not encoded properly leading to the incorrect fitting. \\

After failing at using SciPy, we tried fitting the binned data using Iminuit. Both our own function and ProbFit's inbuilt one's were tried and both of these methods proved unsuccessful at fitting a Crystal Ball to the MC data so they were not pursued further.\\

As a last ditched effort, we tried to use PyROOT and it's \verb|.Fit()|. As seen the Result section, all three models were successfully fitted to the binned data so PyRoot was used for the remainder of the analysis.\\

\addcontentsline{toc}{subsection}{Plot Analysis}
\subsection*{Plot Analysis}
The MC data was fitted with a Gaussian first to get initial guesses of amplitude, mean and standard deviation for the actual data (see figure 11). Then the MC data was fitted using PyRoot's inbuilt Crystal Ball to get values of \textalpha\ and n for the fit of the actual data (see figure 12). This was done because in the actual data, it is not possible to discriminate between the power law tail and the background. \\

The Muon Invariant Mass Pair data was then binned using the Freedman Diaconis method to determine the ideal bin width. The histogram was then fitted using 3 Probability Density functions:
\begin{enumerate}
    \item Sum of three Gaussian's and an exponential
    \item Sum of three double Gaussian's and an exponential
    \item Sum of three Crystal Ball's and an exponential
\end{enumerate}
For the Crystal Ball fit, the values of \textalpha\ and n were not allowed to vary and set to the values found when fitting the MC data.\\

The Gaussian PDF was especially unsuccessful a describing the asymmetry of the \textUpsilon(1s) peak as can be seen in the ratio plot on the lower end of the peak. The Gaussian also failed to properly describe the maxima of the \textUpsilon(1s) peak. The accuracy of the fit over the other peaks was greatly improved as can seen in the histogram and ratio plot.\\

The double Gaussian PDF was more successful than the single Gaussian in describing the data as a whole, especially the maxima of the \textUpsilon(1s) peak. However, it underestimated the low end and overestimated on the high end due to the symmetric nature of the function trying to describe asymmetric data. There was no visible improvement on the fit of the other two peaks.\\

The Crystal Ball PDF was significantly more successful than the Double Gaussian at dealing with the asymmetry of the \textUpsilon(1s) peak. The Crystal Ball PDF was also better at fitting the \textUpsilon(2s) peak as can be seen when comparing the ratio plot's of the double Gaussian and the Crystal Ball. However, its main shortcoming was in describing the amplitude of the \textUpsilon(1s) peak as can be seen in the ratio plot.\\

The reduced $\chi^2$ values of the different fitting models can be seen below:
\begin{table}[H]
\centering
\begin{tabular}{c|c}
Model           & Reduced $\chi^2$           \\ \hline
Gaussian        & 9.976 \\
Double Gaussian & 4.791 \\
Crystal Ball    & 5.003
\end{tabular}
\caption{Reduced \textchi$^2$ values for the different PDF's used to fit the binned data}
\label{tab:my-table}
\end{table}

When taking into account the residual analysis and \textchi$^2$ values in table 1, it was determined that the best PDF to describe the data was the double Gaussian. We will therefore use the Double Gaussian fit to calculate the masses of the \textUpsilon(1s), \textUpsilon(2s) and \textUpsilon(3s) mesons.

\addcontentsline{toc}{subsection}{Systematic Errors}
\subsection*{Errors}

The total error of our systems presented in the paper thus far have been estimated to comprise of 3 parts,

\begin{equation*}
    E_{tot} = \sqrt{E_{stat}^2 + E_{model}^2 + E_{back}^2}
\end{equation*}

$E_{stat}$ is simply the statistical error that is given by PyRoot's \verb|Fit()| function. $E_{model}$ is the systematic error that is given by our lack of knowledge in the data being correctly fitted by any of the functions that were tested. similarly, $E_{Back}$ is the systematic error in our background fit. These errors were estimated by looking at the differences in certain values of results obtained by the fits. So, in the models of the peaks, difference in the number of signal events obtained by the different fits can be compared. This difference can be expressed as a percentage and this percentage is taken to be the systematic error in the model of the peak. Similarly, for the background exponential fit, a good measure of the systematic uncertainty is to fit a linear equation (a polynomial could also be used) to the data, to which we can compare the exponential fit. A good way to compare these two fits would be to look at the area under the minimised curves. The difference in the number of signal events estimated by these two fits is again, a good gauge of the systematic error. From comparing all this data, we calculated the systematic error in the peak model to be $E_{model} =(1- \frac{4536}{4711})\cdot 100 = 3.7\%$ and the systematic error in the background model to be $E_{back} =(1- \frac{8815.088}{8815.111})\cdot 100 = 2.6\cdot 10^{-4}\%$. These errors are used to calculate the final errors in the values of the meson masses. \\

\addcontentsline{toc}{subsection}{Meson Masses}
\subsection*{Meson Masses}

Using the Double Gaussian fit, the masses of the three mesons and the total uncertainty in each of the values was calculated to be 9.4(3), 10.0(3) and 10.3(3) $[GeV/c^2]$.\\